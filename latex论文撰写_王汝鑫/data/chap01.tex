\chapter{绪论}

\section{本模板的意义}

\subsection{现有的学位模板的不足}
学位作为高校教学计划中的重要环节,对提高教学质量、培养学生综合应用能力具有十%
分重要的意义。根据国家制定的《科学技术报告、学位和学术的编写格式》,华南%
师范大学采用Microsoft Word等文字处理系统制作了相应的学位模板。然而,在应用这%
些模板的过程中,却出现了几下几种问题\cite{Maddage2009}:
\begin{enumerate}
\item 无法自动处理图表、公式的序号、参考文献引用标号的变动,经常出现因疏忽而前后%
  文体格式不一致的现象;
\item 无法实现的格式和内容的有效分离,导致模板的格式很容易被学生无意中修改,%
  从而影响了学位规范化管理的质量;
\item 受制于Word版本的兼容性问题,在不同版本的文档之间常会有显示上的出入。%
\end{enumerate}

\subsection{\LaTeX{}的特性}
\TeX{}作为一个功能强大的特别适合于排版科技文献和书籍的格式化排版程序~,其具有以下几点特性:
\begin{enumerate}
\item 强大的宏定义功能。\TeX{}是一种宏命令编程语言,能够处理非常复杂的排版任务,并生成高质量的输出;
\item 方便的自动编号功能。文章、书籍的章、节、段落以及公式、图表、参考文献、页码等均可自动编号;
\item 良好的通用性。\LaTeX{}几乎在所有的计算机操作系统平台上得到实现。\LaTeX{}的源文%
  件可在不同的平台之间自由的交换,而得到的输出是完全相同的。
\end{enumerate}




\subsection{采用\LaTeX{}学位模板的意义}
采用\LaTeX{}学位模板将有以下几点意义:
\begin{enumerate}
\item 有效地解决以往的\verb|Word|模板中所存在的问题,有助于提高毕业生的学位的排版质量;
\item 有助于推动LaTeX在学生群体中的普及,学会使用\LaTeX{}来生成高质量的文档;
\item 本模板在Github\footnote{项目主页:
    \url{http://github.com/wzpan/scnuthesis/}}上开源。让更多爱好者研究学习,并不断完善,以适应日后的需求。
\end{enumerate}


\section{(1.2 题目)}
绪论内容

绪论内容

绪论内容

绪论内容

绪论内容

绪论内容

\section{(1.3 题目)}
绪论内容

绪论内容

绪论内容

绪论内容

绪论内容

绪论内容

\subsection{(1.3.1 题目)}
绪论内容

绪论内容

绪论内容

绪论内容

\subsection{(1.3.2 题目)}

绪论内容

绪论内容


